\documentclass[12pt]{article}    % Specifies the document style.
\usepackage{amsmath,amsfonts}
\usepackage{upgreek}
\usepackage{mleftright}
\usepackage{outlines}
\usepackage[a4paper, total={170mm, 260mm}]{geometry}
\usepackage{hyperref}

\setlength\parindent{0pt}
\setlength{\parskip}{.2cm}

\newcounter{ExNum}
\newcounter{ExMain}
\setcounter{ExMain}{0}
\newcounter{ExMinor}
\setcounter{ExMinor}{0}

\newcommand{\exmajor}[1]{\stepcounter{ExMain}\setcounter{ExMinor}{0}\subsection*{\arabic{ExNum}.\arabic{ExMain} #1}}
\newcommand{\exminor}{\stepcounter{ExMinor}\subsubsection*{Exercise \arabic{ExNum}.\arabic{ExMain}.\arabic{ExMinor}}}
\newcommand{\exminoropt}{\stepcounter{ExMinor}\subsubsection*{Exercise
    \arabic{ExNum}.\arabic{ExMain}.\arabic{ExMinor} (Optional)}}
\newcommand{\exminorhard}{\stepcounter{ExMinor}\subsubsection*{Exercise
    \arabic{ExNum}.\arabic{ExMain}.\arabic{ExMinor} (This is a difficult exercise)}}
\newcommand{\exminorhardopt}{\stepcounter{ExMinor}\subsubsection*{Exercise
    \arabic{ExNum}.\arabic{ExMain}.\arabic{ExMinor} (Optional - this is a difficult exercise)}}
\newcommand{\exthis}{\arabic{ExNum}.\arabic{ExMain}.\arabic{ExMinor}}
\newcommand{\excode}{\texttt{ex\arabic{ExNum}\_\arabic{ExMain}\_\arabic{ExMinor}}}

\newcommand{\exheader}[1]{\section*{02471 Machine Learning for Signal Processing \\ [5mm] Week \arabic{ExNum}: #1}}
\newcommand{\exheadersol}[1]{\section*{02471 Machine Learning for Signal Processing\\[5mm] \textit{Solution}\\[5mm] Exercise \arabic{ExNum}: #1}}
\newcommand{\readingmaterial}[1]{This exercise is based on S. Theodoridis: Machine Learning, A Bayesian and Optimization Perspective 3rd edition,
section(s) #1.}

\newcommand{\xx}{\mathbf{x}}
\newcommand{\yy}{\mathbf{y}}
\newcommand{\Wkm}{W_{k,m}}
%\newcommand{\ss}{\mathbf{s}}
\newcommand{\zz}{\mathbf{z}}
\newcommand{\y}{\bm{y}}
\newcommand{\x}{\bm{x}}
\newcommand{\X}{\text{X}}
\newcommand{\K}{\mathcal{K}}
\newcommand{\s}{\bm{s}}
\newcommand{\z}{\bm{z}}
\newcommand{\R}{\mathbb{R}}
\newcommand{\EE}[1]{\mathbb{E}[#1]}
\newcommand{\E}[1]{\mathbb{E}\mleft[#1\mright]}
\newcommand{\IP}[2]{\langle #1,#2\rangle}
\newcommand{\T}{\bm{\theta}}
\DeclareMathOperator*{\argmin}{arg\,min}
\DeclareMathOperator*{\argmax}{arg\,max}
\def\bo{\begin{outline}[itemize]}
  \def\eo{\end{outline}}
\def\bi{\begin{itemize}}
  \def\ei{\end{itemize}}
\usepackage{graphicx}
\usepackage{bm}

% fix spacing before \left(
% see https://tex.stackexchange.com/questions/2607/spacing-around-left-and-right
\let\originalleft\left
\let\originalright\right
\renewcommand{\left}{\mathopen{}\mathclose\bgroup\originalleft}
\renewcommand{\right}{\aftergroup\egroup\originalright}

\usepackage{times}
\setcounter{ExNum}{13}

\begin{document}
\exheadersol{Diffusion models} 

\exmajor{Diffusion models - Forward process}
\exminor
The following can be written as a markov chain as the each term is only related to the previous:
\begin{equation} \label{eq:basic-diffusion}
    X_t = \sqrt{1-\beta_t} X_{t-1} + \sqrt{\beta_t} \epsilon_t, \quad \epsilon_t \sim \mathcal{N}(0,I)
\end{equation}
For the rewrite we need to use a kernel function, and can thus make the solution:
\begin{equation} \label{eq:markov-diffusion}
    q(X_t|X_{t-1}) = K(X_t | X_{t-1}, \beta_t)
\end{equation}
Where
\begin{equation}
  K(x, b) = \sqrt{1-b} x + \sqrt{b} \epsilon, \quad \epsilon \sim \mathcal{N}(0,I)
\end{equation}

\exminorhard
Start with equation 1 (from the exercise), but with $1-\alpha = \beta$:
\begin{equation}
    X_t = \sqrt{\alpha_t} X_{t-1} + \sqrt{1-\alpha_t} \epsilon_t, \quad \epsilon_t \sim \mathcal{N}(0,I)
\end{equation}
Write out a few steps:
\begin{align}
    X_1 =& \sqrt{\alpha_1} X_0 + \sqrt{1-\alpha_1} \epsilon_1 \\
    X_2 =& \sqrt{\alpha_2} X_1 + \sqrt{1-\alpha_2} \epsilon_2
\end{align}
substitute $X_1$ in $X_2$:
\begin{align}
    X_2 =& \sqrt{\alpha_2} (\sqrt{\alpha_1} X_0 + \sqrt{1-\alpha_1} \epsilon_1) + \sqrt{1-\alpha_2} \epsilon_2 \\
    =& \sqrt{\alpha_2 \alpha_1} X_0  + \sqrt{\alpha_2(1-\alpha_1)} \epsilon_1 + \sqrt{1-\alpha_2} \epsilon_2
\end{align}
By induction, and using the linear property of Gaussians, this gives us:
\begin{equation}
    X_t = \sqrt{\bar{\alpha}_t} + \sum_{s=1}^{t}\left( \sqrt{(1 - \alpha_s) \prod_{j=s+1}^{t}\alpha_j} \epsilon_s \right)
\end{equation}
Where:
\begin{equation}
    \sum_{s=1}^{t}\left((1 - \alpha_s) \prod_{j=s+1}^{t}\alpha_j \right) = 1 - \bar{\alpha}_t
\end{equation}
Which again is proven by induction:
\begin{align}
    t=1 \text{: }& (1 - \alpha_1) = 1 - \bar{\alpha}_1 \\
    t=2 \text{: }& (1 - \alpha_1) \alpha_2 + (1 - \alpha_2) = \alpha_2 - \alpha_1 \alpha_2  + 1 - \alpha_2 = 1 - \alpha_1 \alpha_2 =  1 - \bar{\alpha}_2
\end{align}
This allows us to write the final term as equation 2 (from the exercise):
\begin{equation} 
    X_t = \sqrt{\bar{\alpha}_t}X_0+\sqrt{1-\bar{\alpha}_t} \epsilon_t, \quad \epsilon_t \sim \mathcal{N}(0,I)
\end{equation}

\exmajor{Diffusion models - Reverse process and training}

\exminor
The process can be formalized, using Bayes theorem, using the following:
\begin{equation} \label{eq:denoising-bayes}
    q(X_{t-1}|X_t) = \frac{q(X_t|X_{t-1}) q(X_{t-1})}{q(X_t)}
\end{equation}
While this might seem simple, it will require us to know the distribution, $q(X_0)$, i.e. the distribution of the original data point.
As there are no evidence suggesting the parameters of this distribution, let alone its type, there is no good way of using the model from equation 1 (from the exercise).

\exminor
Start with equation 4 (from the exercise).
From here we would like to split out the terms that affect either $X_0$ or $X_T$ (remember that $\log(a \cdot b) = \log a + \log b$):
\begin{align}
    \log p_{\theta}(X_0) \geq& \mathbb{E}_{q(X_{1:T}|X_0)}\left[ \log p(X_T) + \log \prod_{t=1}^{T}p_{\theta}(X_{t-1}|X_t) - \log \prod_{t=1}^{T} q(X_t|X_{t-1}) \right]\\
    \log p_{\theta}(X_0) \geq&\mathbb{E}_{q(X_{1:T}|X_0)} [\log p_{\theta}(X_0|X_1)] + \\
    &\quad \mathbb{E}_{q(X_{1:T}|X_0)}\left[ \log p(X_T) + \log \prod_{t=2}^{T}p_{\theta}(X_{t-1}|X_t) - \log \prod_{t=1}^{T} q(X_t|X_{t-1}) \right]\\
    \log p_{\theta}(X_0) \geq&\mathbb{E}_{q(X_{1:T}|X_0)} \left[ \log \frac{p(X_T)}{q(X_T|X_{T-1})} \right] + \mathbb{E}_{q(X_{1:T}|X_0)} [\log p_{\theta}(X_0|X_1)] +\\
    &\quad \mathbb{E}_{q(X_{1:T}|X_0)}\left[ \log \prod_{t=2}^{T}p_{\theta}(X_{t-1}|X_t) - \log \prod_{t=1}^{T-1} q(X_t|X_{t-1}) \right]
\end{align}
This will allow us to write up the final expression which can also be found as eq. 19.64 in the book:
\begin{align}
    \log p_{\theta}(X_0) \geq &\underbrace{\mathbb{E}_{q(X_{1}|X_0)} [\log p_{\theta}(X_0|X_1)]}_{\text{reconstruction term}} +\\
    &\underbrace{\mathbb{E}_{q(X_{T-1:T}|X_0)} \left[ \log \frac{p(X_T)}{q(X_T|X_{0})} \right]}_{\text{prior-matching term}} + \\
    &\underbrace{\sum_{t=2}^{T}\mathbb{E}_{q(X_{t}|X_0)}\left[\frac{p_{\theta}(X_{t-1}|X_{t})}{q(X_{t-1}|X_{t})} \right]}_{\text{step-wise denoising term}}
\end{align}

\exminor
As all parts of the equation are known we will simply write the following:
\begin{align}
    q(X_t|X_0) =& \mathcal{N}(X_t| \sqrt{\bar{\alpha}}X_0, (1-\bar{\alpha})\mathbf{I}) \\
    q(X_{t-1}|X_0) =& \mathcal{N}(X_t| \sqrt{\bar{\alpha}_{t-1}}X_0, (1-\bar{\alpha}_{t-1})\mathbf{I})\\
    q(X_t|X_{t-1},X_0) =& \mathcal{N}(X_t| \sqrt{1-\beta_t}X_{t-1}, \beta_t \mathbf{I})
\end{align}

\exminor
We can write out the expression as:
\begin{equation} \label{eq:forward-process-distribution}
    q(X_{t-1}|X_{t}, X_0) = \mathcal{N}(X_{t-1}| \tilde{\mu}_t(X_t, X_0), \tilde{\beta}_t \mathbf{I})
\end{equation}
Where:
\begin{equation*}
    \tilde{\mu}_t(X_t, X_0) = \frac{\sqrt{\bar{\alpha}_{t-1}}\beta_t}{1-\bar{\alpha}_t}X_0 + \frac{\sqrt{\alpha_t}(1-\bar{\alpha}_{t-1})}{1-\bar{\alpha}_t}X_t, \quad \tilde{\beta}_t = \frac{1-\bar{\alpha}_{t-1}}{1-\bar{\alpha}_t}\beta_t
\end{equation*}

\exminorhard
Start by isolating $X_0$ in equation 19.59 in the book.
\begin{equation}
    X_t = \sqrt{\bar{\alpha}_t}X_0 + \sqrt{1-\bar{\alpha}_t}\epsilon_t \Rightarrow X_0 = \frac{1}{\sqrt{\bar{\alpha}_t}}(X_t - \sqrt{1-\bar{\alpha}_t} \epsilon_t)
\end{equation}
Then we substitute that into the mean found in exercise 13.2.4.
\begin{align}
    \tilde{\mu}_p &= \frac{\sqrt{\bar{\alpha}_{t-1}}\beta_t}{1-\bar{\alpha}_t} \frac{1}{\sqrt{\bar{\alpha}_t}} \left(X_t - \sqrt{1-\bar{\alpha}_t} \epsilon_t \right) + \frac{\sqrt{\alpha_t}(1-\bar{\alpha}_{t-1})}{1-\bar{\alpha}_t}X_t \\
    &= \frac{\sqrt{\bar{\alpha}_{t-1}}\beta_t}{1-\bar{\alpha}_t} \left( \frac{X_t}{\sqrt{\bar{\alpha}_t}} - \frac{\sqrt{1-\bar{\alpha}_t} \epsilon_t}{\sqrt{\bar{\alpha}_t}} \right) + \frac{\sqrt{\alpha_t}(1-\bar{\alpha}_{t-1})}{1-\bar{\alpha}_t}X_t
\end{align}
Then use that:
\begin{equation}
    \frac{\sqrt{\bar{\alpha}_{t-1}}}{\sqrt{\bar{\alpha}_t}} = \frac{1}{\sqrt{\alpha_t}}
\end{equation}
To get the following:
\begin{equation}
    \tilde{\mu}_p = \frac{\beta_t}{(1-\bar{\alpha}_t) \sqrt{\alpha_t}} X_t - \frac{\beta_t \sqrt{1-\bar{\alpha}_t} }{(1-\bar{\alpha}_t) \sqrt{\alpha_t}} \epsilon_t + \frac{\sqrt{\alpha_t}(1-\bar{\alpha}_{t-1})}{1-\bar{\alpha}_t}X_t
\end{equation}
Then we collect the $X_t$-terms:
\begin{equation}
    \tilde{\mu}_p = \frac{1}{1-\bar{\alpha}_{t}}\left( \frac{\beta_t}{\sqrt{\alpha_t}} + \sqrt{\alpha_t} (1-\bar{\alpha}_{t-1}) \right) X_t - \frac{\beta_t \sqrt{1-\bar{\alpha}_t} }{(1-\bar{\alpha}_t) \sqrt{\alpha_t}} \epsilon_t
\end{equation}
Multiply the $X_t$-terms with $1 = \frac{\sqrt{\alpha_t}}{\sqrt{\alpha_t}}$
\begin{equation}
    \tilde{\mu}_p = \frac{1}{\sqrt{\alpha_t} (1-\bar{\alpha}_{t})}\left( \beta_t + \alpha_t (1-\bar{\alpha}_{t-1}) \right) X_t - \frac{\beta_t \sqrt{1-\bar{\alpha}_t} }{(1-\bar{\alpha}_t) \sqrt{\alpha_t}} \epsilon_t
\end{equation}
Reduce the parentheses:
\begin{equation}
    \left( \beta_t + \alpha_t (1-\bar{\alpha}_{t-1}) \right) = \left( 1 - \alpha_t + \alpha_t - \bar{\alpha}_{t-1} \alpha_t \right) = 1 - \alpha_t
\end{equation}
And substitute back in:
\begin{equation}
    \tilde{\mu}_p = \frac{1}{\sqrt{\alpha_t}} X_t - \frac{\beta_t \sqrt{1-\bar{\alpha}_t} }{(1-\bar{\alpha}_t) \sqrt{\alpha_t}} \epsilon_t
\end{equation}
And the final term is:
\begin{equation}
    \tilde{\mu}_p = \frac{1}{\sqrt{\alpha_t}} X_t - \frac{1}{\sqrt{\alpha_t}} \frac{\beta_t \sqrt{1-\bar{\alpha}_t} }{(1-\bar{\alpha}_t)} \epsilon_t
\end{equation}


\end{document}
