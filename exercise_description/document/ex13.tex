\documentclass[12pt]{article}    % Specifies the document style.
\usepackage{amsmath,amsfonts}
\usepackage{upgreek}
\usepackage{mleftright}
\usepackage{outlines}
\usepackage[a4paper, total={170mm, 260mm}]{geometry}
\usepackage{hyperref}

\setlength\parindent{0pt}
\setlength{\parskip}{.2cm}

\newcounter{ExNum}
\newcounter{ExMain}
\setcounter{ExMain}{0}
\newcounter{ExMinor}
\setcounter{ExMinor}{0}

\newcommand{\exmajor}[1]{\stepcounter{ExMain}\setcounter{ExMinor}{0}\subsection*{\arabic{ExNum}.\arabic{ExMain} #1}}
\newcommand{\exminor}{\stepcounter{ExMinor}\subsubsection*{Exercise \arabic{ExNum}.\arabic{ExMain}.\arabic{ExMinor}}}
\newcommand{\exminoropt}{\stepcounter{ExMinor}\subsubsection*{Exercise
    \arabic{ExNum}.\arabic{ExMain}.\arabic{ExMinor} (Optional)}}
\newcommand{\exminorhard}{\stepcounter{ExMinor}\subsubsection*{Exercise
    \arabic{ExNum}.\arabic{ExMain}.\arabic{ExMinor} (This is a difficult exercise)}}
\newcommand{\exminorhardopt}{\stepcounter{ExMinor}\subsubsection*{Exercise
    \arabic{ExNum}.\arabic{ExMain}.\arabic{ExMinor} (Optional - this is a difficult exercise)}}
\newcommand{\exthis}{\arabic{ExNum}.\arabic{ExMain}.\arabic{ExMinor}}
\newcommand{\excode}{\texttt{ex\arabic{ExNum}\_\arabic{ExMain}\_\arabic{ExMinor}}}

\newcommand{\exheader}[1]{\section*{02471 Machine Learning for Signal Processing \\ [5mm] Week \arabic{ExNum}: #1}}
\newcommand{\exheadersol}[1]{\section*{02471 Machine Learning for Signal Processing\\[5mm] \textit{Solution}\\[5mm] Exercise \arabic{ExNum}: #1}}
\newcommand{\readingmaterial}[1]{This exercise is based on S. Theodoridis: Machine Learning, A Bayesian and Optimization Perspective 3rd edition,
section(s) #1.}

\newcommand{\xx}{\mathbf{x}}
\newcommand{\yy}{\mathbf{y}}
\newcommand{\Wkm}{W_{k,m}}
%\newcommand{\ss}{\mathbf{s}}
\newcommand{\zz}{\mathbf{z}}
\newcommand{\y}{\bm{y}}
\newcommand{\x}{\bm{x}}
\newcommand{\X}{\text{X}}
\newcommand{\K}{\mathcal{K}}
\newcommand{\s}{\bm{s}}
\newcommand{\z}{\bm{z}}
\newcommand{\R}{\mathbb{R}}
\newcommand{\EE}[1]{\mathbb{E}[#1]}
\newcommand{\E}[1]{\mathbb{E}\mleft[#1\mright]}
\newcommand{\IP}[2]{\langle #1,#2\rangle}
\newcommand{\T}{\bm{\theta}}
\DeclareMathOperator*{\argmin}{arg\,min}
\DeclareMathOperator*{\argmax}{arg\,max}
\def\bo{\begin{outline}[itemize]}
  \def\eo{\end{outline}}
\def\bi{\begin{itemize}}
  \def\ei{\end{itemize}}
\usepackage{graphicx}
\usepackage{bm}

% fix spacing before \left(
% see https://tex.stackexchange.com/questions/2607/spacing-around-left-and-right
\let\originalleft\left
\let\originalright\right
\renewcommand{\left}{\mathopen{}\mathclose\bgroup\originalleft}
\renewcommand{\right}{\aftergroup\egroup\originalright}

\usepackage{times}
\setcounter{ExNum}{13}

\begin{document}
\exheader{Diffusion models} 

\readingmaterial{19.7 - 19.7.2}

The objective of this exercise is to get more familiar with diffusion models for signal processing. We will derive the main computations of the diffusion process, both forward and backward.
Afterwards we will apply a small U-Net structure to a diffusion process to remove noise from the Audio MNIST data set.

\subsection*{Overview}
The exercise have the following structure:

13.1 will take you through the forward process of a diffusion model.

13.2 will have you derive the main mathematical expressions for the reverse process.

13.3 will give you a practical example of a diffusion model.

\subsection*{Notation}
\begin{itemize} 
  \item $X_t$ is a data point a noise-step $t$, where $X_0$ would be a clean data point and $X_T$ would be complete noise.
\end{itemize}

\exmajor{Diffusion models - Forward process}
In exercise you will work through the "noising" process in diffusion. 
We'll derive the necessary expressions for adding noise to data and creating a foundation for removing the noise.
\exminor
We will start with a standard formulation of the noising process in diffusion:
\begin{equation} \label{eq:basic-diffusion}
    X_t = \sqrt{1-\beta_t} X_{t-1} + \sqrt{\beta_t} \epsilon_t, \quad \epsilon_t \sim \mathcal{N}(0,I)
\end{equation} 
Relate this to a markov chain and rewrite the expression to infer $X_t$ directly from $X_0$.

Think about how this adds noise to the data. Is it related to an autoregressive process?

\exminorhard
In diffusion we often like to relate to denoising rather than noising, for this we will introduce two new variables: $\alpha_t = 1 - \beta$ and $\bar{\alpha}_t = \prod_t = \alpha_t$.
Use these variables to rewrite the noising process in terms of $\bar{\alpha}_t$ resulting in (eq. 19.59) from the book:
\begin{equation} \label{eq:forward-process}
    X_t = \sqrt{\bar{\alpha}_t}X_0+\sqrt{1-\bar{\alpha}_t} \epsilon_t, \quad \epsilon_t \sim \mathcal{N}(0,I)
\end{equation}

\exmajor{Diffusion models - Reverse process and training}

\exminor
We will start with the "obvious" way of denoising data, working from the formulations in (eq. 19.56 and 19.57) from the book.

Start by formalizing the denoising process using Bayes theorem:
\begin{equation}
  q(X_{t-1}|X_t) = \dots
\end{equation}
This gives an analytical solution to the denoising process. 
By now, you should realize, that this problem does not have an analytical solution.
Explain why we cannot solve this problem with Bayes theorem.

\exminor
We will instead use the model described in (eq. 19.60 and 19.61) in the book, 
which describes how we can remove noise in an iterative fashion using machine learning.
To use this model we will need a trained predictive model, and as for all trained models we need a loss function. 
For this we will use the Evidence Lower Bound (ELBO) (see chapter 12 in the book).
Using the definition of the ELBO loss and the model described in (eq. 19.60 and 19.61) in the book, we can state the following:
\begin{equation}
    \log p_{\theta}(X_0) \geq \mathbb{E}_{q(X_{1:T}|X_0)}\left[ \log \frac{p(X_T) \prod_{t=1}^{T} p_{\theta}(X_{t-1}|X_t)}{\prod_{t=1}^{T} q(X_t|X_{t-1})} \right]
\end{equation}
In which $\theta$ describes the parameter space for the trained model.

Show that this expression can be split into the three terms (from eq. 19.64 in the book):
\begin{equation}
    \log p_{\theta}(X_0) \geq \underbrace{\mathbb{E}_{q(X_{1}|X_0)} [\log p_{\theta}(X_0|X_1)]}_{\text{reconstruction term}} +\underbrace{\mathbb{E}_{q(X_{T-1:T}|X_0)} \left[ \log \frac{p(X_T)}{q(X_T|X_{0})} \right]}_{\text{prior-matching term}} + \underbrace{\sum_{t=2}^{T}\mathbb{E}_{q(X_{t}|X_0)}\left[\frac{p_{\theta}(X_{t-1}|X_{t})}{q(X_{t-1}|X_{t})} \right]}_{\text{step-wise denoising term}}
\end{equation}

\exminor
We can again describe the method using Bayes theorem:
\begin{equation} \label{eq:bayes-diffusion-denoising}
    q(X_{t-1}|X_{t}, X_0) = \frac{q(X_{t-1}|X_0)q(X_t|X_{t-1},X_0)}{q(X_t|X_0)}
\end{equation} 
But this time all the terms are known. Write out the following:
\begin{align}
    q(X_t|X_0) =& \dots \\
    q(X_{t-1}|X_0) =& \dots \\
    q(X_t|X_{t-1},X_0) =& \dots
\end{align}

\exminor
This will allow us to write out the method in terms of a normal distribution. 
\begin{equation} \label{eq:forward-process-distribution}
    q(X_{t-1}|X_{t}, X_0) = \mathcal{N}(X_{t-1}| \tilde{\mu}_t(X_t, X_0), \tilde{\beta}_t \mathbf{I})
\end{equation}
Use the output of the last exercise to determine the mean and variance of the model:
\begin{align}
  \tilde{\mu}_t(X_t, X_0) &= \dots \\
  \tilde{\beta}_t &= \dots
\end{align}

\exminorhard
As you might have noticed, we do not have an estimate of $X_0$, but we need it to calculate the mean, $\tilde{\mu}_t(X_t, X_0)$.
Again using the expression from the forward pass (eq. 19.59 from the book), rewrite the mean to remove the dependency on $X_0$:
\begin{equation}
  \tilde{\mu}_t(X_t) = \dots
\end{equation}

\exmajor{Training a diffusion model for speech enhancement}

\exminor 
Open the file for this exercise and implement the missing lines. 
The functions should calculate the mean and variance of both the forward and the backward diffusion process.

\exminor
First, download the data from learn and place it in the `week\_13\_python` folder.
 
Run the model by calling the `train.py` file in the `week\_13\_python` folder. This will start a trining of the diffusion model.
When the training is done, a folder `learning\_results` will appear and in it there is a timestamped folder with your newly trained model.

Use the python notebook for this exercise to explore the functionality of the model. Train with different parameters and investigate the output.

\section*{Hints}

Exercise 13.1.2

Is the following true $1-\bar{\alpha}_t = \prod_{n=0}^{t} \beta_n$?

Exercise 13.2.1

Which terms, from Bayes theorem, are known for this system?

Exercise 13.2.3

Use the expression from the forward process (eq. 19.59 from the book) and remember that this is a markov chain.

Exercise 13.2.5

See equation 19.71 in the book

\end{document}
